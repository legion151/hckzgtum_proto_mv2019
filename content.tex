\section{Begrüßung und Formalien}
Florian O. begrüßt alle Anwesenden und eröffnet die Veranstaltung. 

Das Protokoll wird Benedikt C. führen.

Florian O. stellt fest, dass fristgerecht geladen wurde und die Versammlung beschlussfähig ist. 

Er verließt die Tagesordnung, die einstimmig beschlossen wird.

Die Anwesenheitsliste wird erstellt.

Das Protokoll der letzten Versammlung wird auf Antrag nicht verlesen. 


\section{Bericht des Vorstands und des Kassenführers}


\section{Aussprache über die Berichte}
\section{Bericht der Kassenprüfer und Entlastung des Vorstands}

\section{Vorstandswahlen}
 Florian O merkt an, dass Florian K., der zur Sitzung verhindert ist, im Vorfeld schriftlich bereiterklärt hat, im Falle seiner Wahl in einen Vorstandsposten, diesen anzunehmen. (siehe Anhang)
 Florian O. schlägt vor, die Vorsitzenden einzeln zu wählen; die Beisitzer en block mit maximal 3 Durchgängen.
 Linda O. wird von der Versammlung zum Wahlvorstand bestimmt.
\subsection{Wahl des Vorsitzenden}
 Florian O. wird als einziger Kandidat einstimmig bei einer Enthaltung gewählt und nimmt die Wahl an. 
\subsection{Wahl des zweiten Vorsitzenden}         
 Florian Kriebel wird als einziger Kandidat einstimmig gewählt und nimmt die Wahl, wie schriftlich vorher angegeben, an. 
\subsection{Wahl der Beisitzer}
  Vorgeschlagen werden: Jens T., Valentin K. und Florian S.
  Die Wahl wird wie vorgeschlagen durchgeführt. 
  Alle Kandidaten werden einstimmig gewählt und nehmen die Wahl an. 
  Die neuen Beisitzer sind: Florian S., Jens T., Valentin K.

\section{Wahl der Kassenprüfer}
 Tobias W. und Rainer S. und werden beide einstimmig gewählt. 
 
\section{Jahresplanung:   Projekte,   Veranstaltungen, Weiteres}
 
\section{Sonstiges und Unvorhergesehenes}
 Florian O. beendet die Sitzung und weißt auf die anschließende Geburtstagsfeier des Hackzogtums hin. 
