\section{Begrüßung und Formalien}
Florian O. begrüßt alle Anwesenden und eröffnet die Veranstaltung. 

Das Protokoll wird Benedikt C. führen.

Florian O. stellt fest, dass fristgerecht geladen wurde und die Versammlung beschlussfähig ist. 

Er verließt die Tagesordnung, die einstimmig beschlossen wird.

Die Anwesenheitsliste wird erstellt.

Das Protokoll der letzten Versammlung wird auf Antrag nicht verlesen. 


\section{Bericht des Vorstands und des Kassenführers}
Im Anschluss an die letztjährige Mitgliederversammlung fand die Geburtstagsfeier statt, in deren Anschluss der ausgeliehene Cocktailbot mit einem Upgrade versehen wurde.

Ende Juli 2018 fand YouCo statt, wo der Verein einen Stand hatte. Hier wurde das von Rainer erstellte Design der Platinen als auch die Finanzierung der Hardware durch den CCC lobend erwähnt.

Die Regelmäßigen, Monatlich stattfindenden Vorträge mit Themen wie CAD, Arduino oder EAGLE werden lobend erwähnt und es wird um deren Fortführung gebeten.

Des weiteren haben wir am Linux Presentation Day teilgenommen, wo wir sogar besuch hatten.

Außerdem hat ein Team des Hackzogtum Coburg am 1. Coburger Hackaton Code:Rush teilgenommen und gewonnen. Die Teilnehmer werden beglückwünscht und es wird sich bedankt.

Im Dezember haben wir den 35. Caos Communication Congress mit wesentlich mehr Mitgliedern als noch im Vorjahr besucht.

Nach vier langen Jahren hat der Verein nun endlich eine Vereinshaftpflichtversicherung.

Im neuen Jahr gab es dann ein neues Design für die Vereinshomepage.

Der Nebenraum entwickelt sich stetig weiter und es kommen immer wieder neue Veränderungen und Verbesserungen hinzu. Es wird sich bei allen Akteuren bedankt.

Das regelmäßige Plenum mit Planung von Vorträgen und angepeilten Veränderungen wird lobend erwänt und es wird um dessen Fortführung gebeten.

Eine weitere lobende Erwähnung geht an den immer wieder lebendigen Dienstag mit Open-Space-Day, an dem regelmäßig einige Mitglieder anwesend sind.

Wir haben den neuen Coburger Makerspace Creapolis, der in Zusammenarbeit mit der Hochschule und Coburger Firmen entstanden ist, besucht und wollen das eventuell wider holen.

Florian O. bedankt sich bei allen Aktiven Mitgliedern.

Es gibt keine Ergänzungen zu dem Bericht des Vorstandes.

Der Vorstand berichtet, dass der Verein zum Jahresende 36 Mitglieder hatte und im laufenden Jahr ein Austritt sowie vier Eintritte zu vermelden sind.

Der Kassenführer zeigt eine saubere Auflistung der Einnahmen und Ausgaben und berichtet, dass keine GEZ Gebühren bezahlt werden müssen.
Des weiteren wurden die Hardware-Kosten für die YouCo-Lötprojekte vom CCC übernommen.
Die Spende des Preisgeldes an den Verein, die das Team des Hackzogtum Coburg beim 1. Coburger Hackaton Code:Rush gewinnen konnte wird lobend erwähnt.

Es wird erwähnt, dass der Verein ohne die Spende und die Erstattung einen Verlust von ca. 500 € gehabt hätte.
Des weiteren wird berichtet, dass der Stromverbrauch in etwa gleich geblieben ist und etwas weniger Gas gebraucht wurde, jedoch durch Preiserhöhungen die Energiekosten insgesamt gestiegen sind.

Für das laufende Jahr ist anzumerken, dass Kosten für Hardware vom CCC übernommen werden, außerdem gab es eine Hardware-Spende von Watterott gegen Spendenquittung, des weiteren soll der 2. Stromtarif abgemeldet werden. Durch einen Ein- und einen Austritt sind die Mitgliederzahlen gleichgeblieben. Spendenquittungen für Mitgliedsbeiträge können auf Anfrage erhalten werden.

Florian O. bedankt sich bei Jens T. für die Kassenführung.



\section{Aussprache über die Berichte}
Zu den Berichten werden keine Fragen gestellt.

\section{Bericht der Kassenprüfer und Entlastung des Vorstands}
Florian O. bedankt sich bei Jens T. für die Kassenführung und bittet die Kassenprüfer um einen Bericht.

Rainer S. beanstandet die Buchführung der Belege, dass diese nur digital vorliegen.

Es wird Diskutiert, wie die Belege vorliegen müssen, Florian S. merkt an, dass dies in Ordnung ist, wenn die Rechnung digital ist. Rainer S. wird sich mal informieren, wie das wirlich sein muss.

Ansonsten gibt es sowohl von Tobias W., als auch von Rainer S. keine weiteren Beanstandunen und die Kasse ist in Ordnung.

Es wird um eine Entlastung des Vorstandes gebeten.
Dieser Antrag wird einstimmig angenommen.

\section{Vorstandswahlen}
 Florian O merkt an, dass Florian K., der zur Sitzung verhindert ist, sich im Vorfeld schriftlich bereiterklärt hat, im Falle seiner Wahl in einen Vorstandsposten, diesen anzunehmen. (siehe Anhang)
 Florian O. schlägt vor, die Vorsitzenden einzeln zu wählen; die Beisitzer en block mit maximal 3 Durchgängen.
 Linda O. wird von der Versammlung zum Wahlvorstand bestimmt.
\subsection{Wahl des Vorsitzenden}
 Florian O. wird als einziger Kandidat einstimmig bei einer Enthaltung gewählt und nimmt die Wahl an. 
\subsection{Wahl des zweiten Vorsitzenden}         
 Florian Kriebel wird als einziger Kandidat einstimmig gewählt und nimmt die Wahl, wie schriftlich vorher angegeben, an. 
\subsection{Wahl der Beisitzer}
  Vorgeschlagen werden: Jens T., Valentin K. und Florian S.
  Die Wahl wird wie vorgeschlagen durchgeführt. 
  Alle Kandidaten werden einstimmig gewählt und nehmen die Wahl an. 
  Die neuen Beisitzer sind: Florian S., Jens T., Valentin K.

\section{Wahl der Kassenprüfer}
 Tobias W. und Rainer S. und werden beide einstimmig gewählt. 
 
\section{Jahresplanung:   Projekte,   Veranstaltungen, Weiteres}
Ende Juni besuchen besuchen wir das YouCo in Coburg, im August fahren wir nach Brandenburg zum Caos Communication Camp.

Weiterhin bemühen wir uns unsere erfolgreiche Vortragsreihe fortzuführen und andere Hackerspaces zu besuchen.

Es gibt bemühungen erfa zuwerden. 

Wenn es einen Markt der Möglichkeiten gibt werden wir dort ebenfall teilnehmen.

Außerdem haben wir uns vorgenommen, uns am Caos Communication Congress zu treffen.

Es gibt Überlegenungen, ein Gamejam zu planen und durchzuführen, jedoch noch keine kronkreten Pläne.
 
\section{Sonstiges und Unvorhergesehenes}
Jens überlegt, die Stromkosten zu senken durch einen Wechsel des Stromtarifes oder des Anbieters.
Dazu gab es keine Einwände, außer dem Vorschlag bei der SÜC für ein Sponsoring des Internetzugangs anzufragen.

Ein Vorschlag von Florian S. den Termin für die Geburtstagsfeier im nächten Jahr frühzeitig festzulegen wird von Florian O. damit beantwortet, dass man im Januar anfangen werde, nach einem solchen zu suchen.

Florian O. bedankt sich noch für das Aufräumen der Räumlichkeiten im Vorfeld des Geburtstages.

 Florian O. beendet die Sitzung und weißt auf die anschließende Geburtstagsfeier des Hackzogtums hin. 
