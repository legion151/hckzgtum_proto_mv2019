\section{Begrüßung und Formalien}
Florian O. begrüßt alle Anwesenden und eröffnet die Veranstaltung. 

Das Protokoll wird Benedikt C. führen.

Florian O. stellt fest, dass fristgerecht geladen wurde und die Versammlung beschlussfähig ist. 

Er verließt die Tagesordnung, die einstimmig beschlossen wird.

Die Anwesenheitsliste wird erstellt.

Das Protokoll der letzten Versammlung wird auf Antrag nicht verlesen. 


\section{Bericht des Vorstands und des Kassenführers}
Der Vorstand berichtet, dass der Verein zum Jahresende 36 Mitglieder hatte und im laufenden Jahr ein Austritt sowie vier Eintritte zu vermelden sind.

Der Kassenführer zeigt eine saubere Auflistung der Einnahmen und Ausgaben und berichtet, dass keine GEZ Gebühren bezahlt werden müssen.
Des weiteren wurden die Hardware-Kosten für die YouCo-Lötprojekte vom CCC übernommen.
Die Spende des Preisgeldes an den Verein, die das Team des Hackzogtum Coburg beim 1. Coburger Hackaton Code:Rush gewinnen konnte wird lobend erwähnt.

Es wird erwähnt, dass der Verein ohne die Spende und die Erstattung einen Verlust von ca. 500 € gehabt hätte.
Des weiteren wird berichtet, dass der Stromverbrauch in etwa gleich geblieben ist und etwas weniger Gas gebraucht wurde, jedoch durch Preiserhöhungen die Energiekosten insgesamt gestiegen sind.

Für das laufende Jahr ist anzumerken, dass Kosten für Hardware vom CCC übernommen werden, außerdem gab es eine Hardware-Spende von Watterott gegen Spendenquittung, des weiteren soll der 2. Stromtarif abgemeldet werden. Durch einen Ein- und einen Austritt sind die Mitgliederzahlen gleichgeblieben. Spendenquittungen für Mitgliedsbeiträge können auf Anfrage erhalten werden.

Florian O. bedankt sich bei Jens T. für die Kassenführung.



\section{Aussprache über die Berichte}
Zu den Berichten werden keine Fragen gestellt.

\section{Bericht der Kassenprüfer und Entlastung des Vorstands}
Florian O. bittet die Kassenprüfer um ihren Bericht. 

\section{Vorstandswahlen}
 Florian O merkt an, dass Florian K., der zur Sitzung verhindert ist, sich im Vorfeld schriftlich bereiterklärt hat, im Falle seiner Wahl in einen Vorstandsposten, diesen anzunehmen. (siehe Anhang)
 Florian O. schlägt vor, die Vorsitzenden einzeln zu wählen; die Beisitzer en block mit maximal 3 Durchgängen.
 Linda O. wird von der Versammlung zum Wahlvorstand bestimmt.
\subsection{Wahl des Vorsitzenden}
 Florian O. wird als einziger Kandidat einstimmig bei einer Enthaltung gewählt und nimmt die Wahl an. 
\subsection{Wahl des zweiten Vorsitzenden}         
 Florian Kriebel wird als einziger Kandidat einstimmig gewählt und nimmt die Wahl, wie schriftlich vorher angegeben, an. 
\subsection{Wahl der Beisitzer}
  Vorgeschlagen werden: Jens T., Valentin K. und Florian S.
  Die Wahl wird wie vorgeschlagen durchgeführt. 
  Alle Kandidaten werden einstimmig gewählt und nehmen die Wahl an. 
  Die neuen Beisitzer sind: Florian S., Jens T., Valentin K.

\section{Wahl der Kassenprüfer}
 Tobias W. und Rainer S. und werden beide einstimmig gewählt. 
 
\section{Jahresplanung:   Projekte,   Veranstaltungen, Weiteres}
 
\section{Sonstiges und Unvorhergesehenes}
 Florian O. beendet die Sitzung und weißt auf die anschließende Geburtstagsfeier des Hackzogtums hin. 
